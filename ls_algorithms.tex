\documentclass[10pt]{article}
\usepackage[left=2.54cm,top=2.54cm,right=2.54cm,bottom=2.54cm]{geometry}
\usepackage{acronym}
\usepackage{afterpage}
\usepackage{algpseudocode}
\usepackage{algorithm}
\usepackage{bibspacing}
\usepackage{bm}
\usepackage{comment}
\usepackage{fancyhdr}
\usepackage{float}
\usepackage[inline]{enumitem}
\usepackage{sectsty}
\usepackage{setspace}
\usepackage{url}
\usepackage{subcaption}
\usepackage{nicefrac}
\usepackage{units}
\usepackage{booktabs}
\usepackage{xcolor}
\begin{document}
General conservation laws that govern physical systems, when
discretized in space and time and cast in iterative form almost
always result in sparse linear system of form:
\begin{equation}
  A \Delta q = B
\end{equation}
One of the techniques to iteratively solve the system is to use a
Gauss-Seidel preconditioner. One multi-threaded systems, one will
have to create a color order to create a parallel iterative
technique. Coloring involves grouping the variables in X, such that
for every row, the group associated with the degree of freedom
corresponding to an off-diagonal entry is different from
group associated with the diagonal degree of freedom.
Once, coloring has been established, the system above can be
represented as
\begin{equation}
  (L+D+U) \Delta q = B
\end{equation}
where $L$ corresponds to off-diagonal elements of a smaller color
compared to that of the diagonal entry, $D$ is the diagonal entry
and $U$ corresponds to off-diagonal elements of a larger color.
Given these, the Gauss-Seidel algorithm can be written as follows:
\begin{algorithm}
\caption{Exawind driver}
\label{alg:driver}
\begin{algorithmic}[1]

\State Create MPI sub-communicators near-body-comm and off-body-comm
\If{rank $\subset$ near-body-comm}
        \State Initialize near-body solver \Comment{Multiple instances of near-body solver allowed}
\EndIf

\If{rank  $\subset$ off-body-comm}
    \State Initialize off-body solver \Comment{Only single instance allowed}
\EndIf

\State Initialize TIOGA \Comment{For all ranks}
\For{$t = 0; t < T; t \leftarrow t+1$} \Comment{Where T is the total number of time steps}
        \If{near-body mesh movement $\|$ off-body mesh adaption $\|$ $t == 0$}
        \State Perform overset connectivity
        \EndIf
        \State Exchange overset solution
         \For{$p = 0; p < P; p \leftarrow p+1$} \Comment{Where P is the total number of Picard iterations}
        \State Perform near-body time step
     \EndFor
     \State Perform off-body time step \Comment{Executed in parallel to the near-body time step}

        \If{additional Picards}
        \State Exchange overset solution
        \For{$p = 0; p < P; p \leftarrow p+1$} \Comment{Where P is the total number of Picard iterations}
                \State Perform near-body time step
        \EndFor
        \EndIf

\EndFor
\end{algorithmic}
\end{algorithm}


\begin{algorithm}
  \caption{Colored Gauss Seidel Algorithm}
  \begin{algorithmic}[1]	
  \For{$i = 0; i < niter; i \leftarrow i+1$} \comment {Iteration loop}
  %\For{$color=0;$color<ncolors;color \leftarray color+1$}
  %\Delta x = D^{-1}(B[i]-L*dq_{c < color}-U*dq_{c>color})
  %\EndFor
  %\For{$color=ncolors-1;$color>=ncolors;color \leftarray color-1$}
  %\Delta x = D^{-1}(B[i]-L*dq_{c < color}-U*dq_{c>color})
  %\EndFor
  %\EndFor
  \end{algorithmic}
\end{algorithm}

\end{document}
  
